% !Tex program = xelatex
\documentclass{article}
\usepackage{amsmath}
\usepackage[slantfont,boldfont]{xeCJK}
\usepackage{fontspec}
\setCJKmainfont{SimSun}
\setmainfont{SimSun}
\setsansfont{SimSun}
\begin{document}

\section*{方向导数(Directional Derivative)}
方向导数描述了函数在给定方向上的变化率。对于标量场 $f(x, y, z)$,在单位向量 $\mathbf{u} = (u_1, u_2, u_3)$ 方向上的方向导数定义为:
\[
D_{\mathbf{u}} f = \nabla f \cdot \mathbf{u} = u_1 \frac{\partial f}{\partial x} + u_2 \frac{\partial f}{\partial y} + u_3 \frac{\partial f}{\partial z},
\]
其中 $\nabla f$ 是 $f$ 的梯度。

\section*{梯度(Gradient)}
梯度是标量场 $f(x, y, z)$ 的一个向量场,表示函数在各方向上变化率的最大值,其定义为:
\[
\nabla f = \left( \frac{\partial f}{\partial x}, \frac{\partial f}{\partial y}, \frac{\partial f}{\partial z} \right).
\]

\section*{散度(Divergence)}
散度是向量场 $\mathbf{F} = (F_1, F_2, F_3)$ 的一个标量场,表示向量场在某点的“发散程度”,其定义为:
\[
\nabla \cdot \mathbf{F} = \frac{\partial F_1}{\partial x} + \frac{\partial F_2}{\partial y} + \frac{\partial F_3}{\partial z}.
\]

\section*{旋度(Curl)}
旋度是向量场 $\mathbf{F} = (F_1, F_2, F_3)$ 的一个向量场,表示向量场的“旋转程度”,其定义为:
\[
\nabla \times \mathbf{F} = 
\begin{vmatrix}
\mathbf{i} & \mathbf{j} & \mathbf{k} \\
\frac{\partial}{\partial x} & \frac{\partial}{\partial y} & \frac{\partial}{\partial z} \\
F_1 & F_2 & F_3
\end{vmatrix},
\]
其中 $\mathbf{i}, \mathbf{j}, \mathbf{k}$ 是标准基向量,结果为:
\[
\nabla \times \mathbf{F} = \left( 
\frac{\partial F_3}{\partial y} - \frac{\partial F_2}{\partial z}, 
\frac{\partial F_1}{\partial z} - \frac{\partial F_3}{\partial x}, 
\frac{\partial F_2}{\partial x} - \frac{\partial F_1}{\partial y}
\right).
\]

\end{document}
